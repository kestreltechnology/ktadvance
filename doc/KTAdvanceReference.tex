\documentclass[11pt]{article}
\usepackage{epsfig}
\usepackage{subfigure}
\usepackage{rotating} 
\usepackage{amsfonts,amssymb}
\usepackage{amsmath}
\usepackage{graphicx}
\usepackage[longnamesfirst,numbers]{natbib}
\usepackage{setspace}
\usepackage{pifont}
\usepackage{enumitem}
\usepackage{listings} 
\usepackage{wrapfig}

%\setlist{nolistsep,noitemsep,topsep=0pt,parsep=0pt,partopsep=0pt}
\setlist{nolistsep,itemsep=4pt,topsep=4pt,parsep=0pt,partopsep=0pt}
\DeclareGraphicsExtensions{.png,.jpg}
\graphicspath{{.}}

\newcommand\commentout[1]{}
\newcommand\cvt{{\sc cvt}}

\def\Implies{\;\Longrightarrow\;}
\def\Equiv{\;\equiv\;}
\def\Equals{\;\;{\bf = }\;\;}
\def\And{\;\wedge\;}
\def\Or{\;\vee\;}

\headheight 0in
\parindent=0.0in
\parskip=2ex plus1ex
\topmargin 0.0in
\textheight 8.0in
\textwidth 6.250in
\evensidemargin -2.0mm
\oddsidemargin 4.0mm
%\evensidemargin 0.375in
%\oddsidemargin -0.03125in

\usepackage{fancyheadings}
\pagestyle{fancy}
\lhead{KT Advance C Analyzer}
\chead{}
\rhead{Reference Manual}
\cfoot{Kestrel Technology LLC}

\newcommand\ch{{\bf CodeHawk}}
\newcommand\fname{$<$f$>$}
\newcommand\fnname{$<$ff$>$}

%\setcounter{page}{3}

\normalsize
\setlength{\headheight}{2\baselineskip}

\begin{document}

\vfill
\vfill


{\Large\bf{
KT Advance Reference Manual}}

\vfill

{\Large{Kestrel Technology, LLC}}

\bigskip

November 24, 2017

\vfill
\vfill
\vfill

\newpage
\tableofcontents
\newpage

\section{Overview}

The KT Advance C Analyzer consists of three components:
\begin{enumerate}
\item {\bf Parser:} A Mac/Linux executable that takes as input a preprocessed
C source file and produces a set of xml files that precisely represent the
semantics of the C source file;
\item {\bf C Analyzer:} A Mac/Linux executable that takes as input the
semantics files produced by the parser as well as analysis results files if
available, and produces a set of xml files that hold analysis results. The
C Analyzer will be wrapped in a license manager to protect the contained 
intellectual property.
\item {\bf PyAdvance:} Python code, provided as source code to licensed
users, that performs linking and provides various analyzer invocation, 
integration, and reporting services.
\end{enumerate}

\section{Approach}

\subsection{Proof by Induction}

The goal of the KT Advance C Analyzer is to mathematically prove absence
of memory safety vulnerabilities, or, more precisely, prove absence of
undefined behavior related to memory safety. The technique used to accomplish 
this is proof
by structural induction on the control flow graph of each function: for each
instruction we assume that the state of an execution (viewed as a sequence of states) 
reaching this point is well-defined
(the inductive hypothesis), and we have to prove that all possible states
reached after the instruction are still well defined (the inductive step).
If indeed we can show that every instruction, starting from a well-defined
state, ends in a well-defined state, we can conclude that no undefined
behavior is possible in any execution of the application.

\subsection{Primary Proof Obligations}

The inductive step requires proving for each instruction that the weakest
precondition of that instruction with respect to well-definedness as
defined in the C Standard~\cite{cstandard} is program-valid, that is,
it is valid on all possible states that an execution can be in when
reaching that instruction. We have chosen to express these weakest
preconditions as conjunctions of a collection of primitive predicates rather than as one
monolithic predicate to facilitate the use of different analysis domains
and potentially specialized proof tools for different predicates.
We call the primitive predicates that form the weakest precondition
``primary proof obligations.'' For example, the primary proof obligations
(also referred to as ppo's) for 
the instruction {\tt j = j/i} would be that {\tt j} and
{\tt i} both be initialized and that {\tt i} is not equal to zero,
or, formally $initialized(i) \wedge initialized(j) \wedge not-zero(i)$,
all of which can be proven separately. A list and detailed description
of all primitive predicates used is given below.

\subsection{Proof Obligation Predicates}

\paragraph{allocation-base (ptr:exp): }
The value of the ptr expression is the address of a 
                dynamically allocated memory region.

\paragraph{common-base (ptr1:exp, ptr2:exp): }
The value of the ptr1 expression and the value of the ptr2 expression
   point to the same memory region (where a region can be an allocated region, or a region
   defined by a declared variable).
   
\paragraph{common-base-type (ptr1:exp, ptr2:exp): }
The value of the ptr1 expression and the value of the ptr2 expression
  point at elements of the same array.

\paragraph{index-lower-bound (index:exp): }
The value of the index expression is greater than or equal to zero.

\paragraph{index-upper-bound (index:exp, size:exp):}
The value of the index expression is less than the value of the
   size expression. 

\paragraph{initialized (lhs:lval): }
The value in the location denoted by lhs is initialized.

\paragraph{initialized-range (ptr:exp, len:exp): }
The value of the ptr expression points to a memory region of which at least
len bytes are initialized starting from the address in ptr.

\paragraph{lower-bound (ptr:exp):}
The value of the ptr expression is greater than or equal to the
    lower bound of the memory region it is pointing at (vacuously true for NULL).
    
\paragraph{non-negative (scalar:exp):}
The value of the scalar expression is non-negative.

\paragraph{no-overlap (ptr1:exp, ptr2:exp): }
the value of the ptr1 expression and the value of the ptr2
   expression do not point at the same memory region.

\paragraph{not-null (ptr:exp):}
The value of the ptr expression is not NULL.

\paragraph{not-zero (scalar:exp): }
The value of the scalar expression is not zero.

\paragraph{null (ptr:exp): }
The value of the ptr expression is NULL.

\paragraph{null-terminated (ptr:exp): }
The ptr expression points at a memory region that contains a null-terminator
within its bounds.

\paragraph{ptr-lower-bound (t:typ, op:binop, ptr:exp, scalar:exp)}
The result of the operation op performed on the ptr expression and scalar expression
is greater than or equal to the lower bound of the memory region pointed to by
the ptr expression.

\paragraph{ptr-upper-bound (t:typ, op:binop, ptr:exp, scalar:exp)}
The result of the operation op performed on the ptr expression and scalar expression
is less than or equal to the upper bound of the memory region pointed to by the ptr
expression.

\paragraph{upper-bound (ptr:exp):}
The value of the ptr expression is less than or equal to the
    upper bound of the memory region it is is pointing at (vacuously true for NULL).

\paragraph{valid-mem (ptr:exp): }
The value of the ptr expression is the address of or inside a 
	valid memory region, that is, a memory region that has not been freed and
	has not gone out of scope.

The following predicates are not a weakest precondition for undefined behavior, but
are included to identify constructs generally considered undesirable.

\paragraph{format-string (ptr:exp): }
The value of the ptr expression points at a string literal.


\section{Files}

Analysis is modular: the C Analyzer analyzes (preprocessed) c source files
in isolation; the PyAdvance integrator transfers results from one file to
another in terms of api assumptions, postcondition guarantees, etc. These
analysis artifacts are saved in xml files associated with the entire
application, individual files or single functions.
Below we list these files, along with their role in the analysis and their
format and contents.

All analysis artifacts are kept in a subdirectory {\tt semantics/ktadvance}
of the analysis directory.
Below we will refer to this directory as the top directory.

\subsection{Application-level}

The following two files combine information about the entire application
at the top level of the 

\begin{itemize}[leftmargin=*]
\item {\bf target\_files.xml: } a list of the source files included in the
  application. \\
  \emph{created by:} Parser \\
  \emph{updated by:} none
\item {\bf globaldefinitions.xml:} a dictionary of global definitions and declarations
  that are shared by all source files. \\
  \emph{created by:} PyAdvance Linker  (advance/linker) \\
  \emph{updated by:} none
\end{itemize}

\subsection{File-level}

The following files are kept for each source file \fname.c in a directory relative
to the top directory that corresponds to their location in the original application
source directory.
\begin{itemize}[leftmargin=*]
\item {\bf \fname\_cfile.xml: } Global definitions.\\
   \emph{created by:} Parser \\
   \emph{updated by:} none
\item {\bf \fname\_cdict.xml: } Dictionary of types, variables, expressions, etc.
   that appear in the program.\\
   \emph{created by:} Parser \\
   \emph{updated by:} C Analyzer (primary proof obligation generation)
\item {\bf \fname\_gxrefs.xml: } Mapping between global indices and file-local
   indices for struct definitions and global variables. \\
   \emph{created by:} PyAdvance Linker \\
   \emph{updated by:} none
\item {\bf \fname\_ctxt.xml:} Dictionary of precise locations in the program,
   expressed as program contexts, a pair of a cfg-context, specifying the location
   in terms of control-flow-graph nodes, and an exp-context, specifying a location
   within an expression, in terms of nodes in the syntax tree. \\
   \emph{created by:} C Analyzer (primary proof obligation generation). \\
   \emph{updated by:} none
\item {\bf \fname\_ixf.xml:} Dictionary of components of interface expressions,
   such as function preconditions, postconditions, and side effects.\\
   \emph{created by:} C Analyzer (invariant generation) \\
   \emph{updated by:} C Analyzer (invariant generation)
\item {\bf \fname\_prd.xml:} Dictionary of predicates used in primary and supporting
   proof obligations.\\
   \emph{created by:} C Analyzer (primary proof obligation generator) \\
   \emph{updated by:} PyAdvance (creation of supporting proof obligations)
\end{itemize}

\subsection{Function-level}

The following files are kept for each function \fnname in file \fname.c, 
in a subdirectory \fname in the directory that holds the \fname\_xxx.xml
files.

\begin{itemize}[leftmargin=*]
\item {\bf \fnname\_cfun.xml:} Complete semantics of the function, expressed
   as CIL-like data structures, using dictionary indices for types, expressions,
   etc.\\
   \emph{created by:} Parser \\
   \emph{updated by:} none
\item {\bf \fnname\_api.xml:} Application interface artifacts for a function,
  including assumptions on arguments (used to create supporting proof obligations),
  postcondition guarantees provided by the function, postcondition requests from
  other functions.\\
  \emph{created by:} C Analyzer (primary proof obligation generation) \\
  \emph{updated by:} C Analyzer (invariant gernation, proof obligation check), PyAdvance
\item {\bf \fnname\_ppo.xml:} Primary proof obligations for a function. \\
  \emph{created by:} C Analyzer (primary proof obligation generation) \\
  \emph{updated by:} C Analyzer (proof obligation check)
\item {\bf \fnname\_spo.xml:} Supporting proof obligations for a function. \\
  \emph{created by:} C Analyzer (primary proof obligation generation) \\
  \emph{updated by:} PyAdvance, C Analyzer (proof obligation check)
\item {\bf \fnname\_pod.xml:} Dictionary of primary and supporting proof obligation
   types, using predicate and type/expression indices from \fname\_cdict.xml and
   \fname\_prd.xml. \\
   \emph{created by:} C Analyzer (primary proof obligation generation) \\
   \emph{updated by:} PyAdvance (creation of supporting proof obligations)
\item {\bf \fnname\_vars.xml:} Dictionary of analysis artifacts referenced in invariants.\\
   \emph{created by:} C Analyzer (invariant generation) \\
   \emph{updated by:} C Analyzer (invariant generation)
\item {\bf \fnname\_invs.xml:} Dictionary of invariant values and location invariant table.\\
   \emph{created by:} C Analyzer (invariant generation) \\
   \emph{updated by:} C Analyzer (invariant generation)
\end{itemize}

\section{Dictionary Data Structure Formats}

Several file-level and function-level files provide an index representation
of commonly used entities such as expressions and locations. Each data item is
represented by a list of strings and a list of integers, which can be indices
into other dictionaries or immediate values. Each dictionary file generally
has multiple tables for related data structures. Below we describe the
data elements contained in these structures. Indices into other tables are
referred to by the table name.

\subsection{\fname\_cdict}

The primary dictionary file for a source file is the \fname\_cdict.xml file,
which includes entries for all entities relevant to the source code in that
file, including types and expressions. Below we describe each of the tables
included in this file, and for each of the tables a reference to the relevant
python files in PyAdvance that provide the data structures for the entities
in that table. Basic indexing and access to the {\tt \_cdict} dictionary is 
provided by {\tt advance/app/CDictionary.py} Table~\ref{tab:cdict} lists all
tables included in this dictionary and the PyAdvance.py files that implement
the corresponding objects.

\begin{table}
\centering
\begin{tabular}{l|l|l}
name & description & AdvancePy reference\\  \hline
{\tt attr} & program type attributes & app/CAttributes.py \\
{\tt const} & program constants & app/CConstExp.py\\
{\tt exp} & program (sub)expressions & app/CExp.py \\
{\tt host} & program location & app/CLHost.py \\
{\tt lval} & program left-hand values & app/CLval.py\\
{\tt offset} & location offsets (field, index) & app/COffsetExp.py \\
{\tt typ} & program types & app/CTyp.py \\
\end{tabular}
\caption{\label{tab:cdict}Tables included in the \_cdict dictionary}
\end{table}

\subsubsection{Attribute Parameter Table}

The attribute parameter table, shown in Table~\ref{tab:attrtable} contains attributes 
that are associated with types in the program.

\begin{table}
\centering
\begin{tabular}{l|c||c|c|c|c||l}
\multicolumn{2}{c||}{tags} & \multicolumn{4}{|c||}{args} & description \\ \hline
\multicolumn{1}{c|}{1} & 2 & 1 & 2 & 3 & 4 & \\ \hline 
"aint" & & value & & & & integer value \\
"astr" & & {\tt string} & & & & string value \\
"acons" & name & & & & & \\
"asizeof" & & {\tt typ} & & & & \\
"asizeofe" & & {\tt attr} & & & & \\
"asizeofs" & & {\tt typsig} & & & & \\
"aalignof" & & {\tt typ} & & & & \\
"aalignofe" & & {\tt attr} & & & & \\
"aalignofs" & & {\tt typsig} & & & & \\
"aunop" & unop & {\tt attr} & & & & \\
"abinop" & binop & {\tt attr} & {\tt attr} & & & \\
"adot" & & {\tt attr} & & & & \\
"astar" & & {\tt attr} & & & & \\
"aaddrof" & & {\tt attr} & & & & \\
"aindex" & & {\tt attr} & {\tt attr} & & & \\
"aquestion" & & {\tt attr} & {\tt attr} & {\tt attr} & & \\ \hline
\end{tabular}
\caption{\label{tab:attrtable}Attribute-table: Tags and arguments for program type attributes}
\end{table}



\subsubsection{Constant Table}

The constant table, shown in Table~\ref{tab:consttable} contains constants that 
appear in the program. Integer and 
real constants are represented as strings as they appear textually in the program.

\begin{table}
\centering
\begin{tabular}{l|c|c||c|c||l}
\multicolumn{3}{c||}{tags} & \multicolumn{2}{|c||}{args} & description \\ \hline
\multicolumn{1}{c|}{1} & 2 & 3 & 1 & 2 & \\ \hline 
"chr" & & & ord(chr) & & character represented by its ordinal number \\
"int" & string-rep & ikind & & & integer constant \\
"str" & & {\tt string} & & & constant string \\
"wstr" & & & *value & & wide string (represented by list of integers) \\
"real" & string-rep & fkind & & & real number constant \\
\end{tabular}
\caption{\label{tab:consttable}Const-table: Tags and arguments for program constants}
\end{table}


\subsubsection{Expression Table}

The expression table contains expressions and subexpressions that represent
expressions in the program. 
Expressions are represented by an indicator tag followed by an optional second tag
and one through four table indices or immediate values, as shown in
Table~\ref{tab:exptable}. 

The string values for \em{unop} and \em{binop} are shown
in Tables~\ref{tab:unop} and \ref{tab:binop}, respectively.
\begin{table}
\centering
\begin{tabular}{l|c||c|c|c|c||l}
\multicolumn{2}{c||}{tags} & \multicolumn{4}{|c||}{args} & description \\ \hline
\multicolumn{1}{c|}{1} & 2 & 1 & 2 & 3 & 4 & \\ \hline 
"addrof" & & {\tt lval} & & & & address of location \\
"addroflabel" & & stmt index & & & \\
"alignof" & & {\tt typ} & & & & alignment of a type \\
"alignofe" & & {\tt exp} & & & & alignment of type of expr \\
"binop" & binop & {\tt exp} & {\tt exp} & {\tt typ} & & binary expression \\
"caste" & & {\tt typ} & {\tt exp} & & & cast of expression \\
"cnapp" & name & *{\tt exp} & & & & custom predicate on expressions \\
"const" & & {\tt const} & & & & constant \\
"fnapp" & filename & line-nr & byte-nr & {\tt exp} & *{\tt exp} & function application @loc \\
"lval" & & {\tt lval} & & & & left-hand-side (location) \\
"question" & & {\tt exp} & {\tt exp} & {\tt exp} & {\tt typ} & ? : expression \\
"sizeof" & & {\tt typ} & & & & size of type \\
"sizeofe" & & {\tt exp} & & & & size of type of expr \\
"sizeofstr" & & {\tt string} & & & & size of string\\
"startof" & & {\tt lval} & & & & same as addrof \\
"unop" & unop & {\tt exp} & {\tt typ} & & & unary expression \\ \hline
\end{tabular}
\caption{\label{tab:exptable}Exp-table: Tags and arguments for expressions}
\end{table}



\begin{table}[t]
\centering
\begin{tabular}{l|l} \hline
"neg" & arithmetic negation \\
"bnot" & bitwise complementation \\
"lnot" & logical not  \\ \hline
\end{tabular}
\caption{\label{tab:unop}Unary operators}
\end{table}


\begin{table}[t]
\centering
\begin{tabular}{l|l} \hline
"plusa" & scalar addition \\
"pluspi" & pointer plus scalar \\
"indexpi" & pointer plus scalar\\
"minusa" & scalar subtraction \\
"minuspi" & pointer minus scalar \\
"minuspp" & pointer subtraction \\
"mult" & scalar multiplication \\
"div" & scalar division \\
"mod" & scalar modulo \\
"shiftlt" & bitwise leftshift \\
"shiftrt" & bitwise rightshift \\
"lt" & less than \\
"gt" & greater than \\
"le" & less than or equal to \\
"ge" & greater than or equal to \\
"eq" & equal \\
"ne" & not equal \\
"band" & bitwise and \\
"bxor" & bitwise xor \\
"bor" & bitwise or \\
"land" & logical and \\
"lor" & logical or
\end{tabular}
\caption{\label{tab:binop}Binary operators}
\end{table}

\subsubsection{Lval Table}

An lval (a value that can be on the left side of an assignment, that is, a
program storage location) consists of a base location called a host, and an
optional offset, which can be a field or index. Their representations are
shown in Table~\ref{tab:lvaltable}.

\begin{table}
\centering
\begin{tabular}{l|l|c||c|c|c|c||l}
& \multicolumn{2}{c||}{tags} & \multicolumn{4}{|c||}{args} & description \\ \hline
& \multicolumn{1}{c|}{1} & 2 & 1 & 2 & 3 & 4 & \\ \hline 
\multicolumn{8}{l}{{\tt lval}} \\ \hline
& & & {\tt host} & {\tt offset} & & & left-hand side value \\ \hline
\multicolumn{8}{l}{{\tt host}} \\ \hline
& "var" & name & id & & & & program variable \\
& "mem" & & {\tt exp} & & & & expression dereference \\ \hline
\multicolumn{8}{l}{{\tt offset}} \\ \hline
& "n" & & & & & & no offset \\
& "f" & name & key & {\tt offset} & & & field offset (fieldname,struct identifier) \\
& "i" & & {\tt exp} & {\tt offset} & & & index offset \\ \hline
\end{tabular}
\caption{\label{tab:lvaltable}Lval-table: tags and arguments for program locations}
\end{table}


\subsubsection{Type Table}

The typ-table contains data types that represent types that appear in the program
with elements shown in Table~\ref{tab:typtable}.
Some elements are optional and can be absent; these are shown between brackets
(e.g., {\tt (attrs)}. Some elements are optional and have value -1 if not provided;
these are shown in square brackets (e.g., {\tt [exp]} for the array size expression).

The type of integer is specified by the string indicated as \emph{ikind}, which can 
have the following values with the obvious meanings: "ichar", "ischar", "iuchar","ibool",
   "iint", "iuint", "ishort", "iushort", "ilong", "iulong", "ilonglong", "iulonglong".
The type of float is specified by the string indeciate as \emph{fkind} which can 
have one of the following three values: "float", "fdouble", "flongdouble".

\begin{table}
\centering
\begin{tabular}{l|c||c|c|c|c||l}
\multicolumn{2}{c||}{tags} & \multicolumn{4}{|c||}{args} & description \\ \hline
\multicolumn{1}{c|}{1} & 2 & 1 & 2 & 3 & 4 & \\ \hline 
"tarray" & & {\tt typ} & {\tt [exp]} & {\tt (attrs)} & & array (element type, size) \\
"tbuiltin-va-list" & & {\tt (attrs)} & & & & builtin varargs \\
"tcomp" & & key & {\tt (attrs)} & & & struct/union type identifier \\
"tenum" & name & {\tt (attrs)} & & & & enum type \\
"tfloat" & fkind & {\tt (attrs)} & & & & floating point number types \\
"tfun" & & {\tt typ} & {\tt [fun-args]} & is-varargs & {\tt (attrs)} & function (return type, args) \\
"tint" & ikind & {\tt (attrs)}  & & & & integer types \\
"tnamed" & name & {\tt (attrs)} & & & & type name \\
"tptr" & & {\tt typ} & {\tt (attrs)} & & & pointer to another type \\
"tvoid" & & {\tt (attrs)}  & & & & void type \\ \hline
\end{tabular}
\caption{\label{tab:typtable}Typ-table: tags and arguments for types}
\end{table}

\subsubsection{Type Signature Table}

\begin{table}
\centering
\begin{tabular}{l|c||c|c|c|c||l}
\multicolumn{2}{c||}{tags} & \multicolumn{4}{|c||}{args} & description \\ \hline
\multicolumn{1}{c|}{1} & 2 & 1 & 2 & 3 & 4 & \\ \hline 
"tsarray" & length & {\tt typsig} & {\tt (attrs)} & & & \\
"tsptr" & & {\tt typsig} & {\tt (attrs)} & & & \\
"tscomp" & name & is-struct & {\tt (attrs)} & & \\
"tsfun" & & {\tt typsig} & {\tt typsig-list} & is-varargs & {\tt (attrs)} & \\
"tsenum" & name & {\tt (attrs)} & & & & \\
"tsbase" & & {\tt typsig} & & & &
\end{tabular}
\caption{\label{tab:typtable}Typsig-table: tags and arguments for type signatures}
\end{table}

\subsection{Declarations Tables}

Table~\ref{tab:decls} shows the data representation of the various types of 
declarations included in the program. The compinfo represents struct and union
definitions. The enuminfo represents enum definitions. The init represents
initializers for global variables. The typeinfo represents type definitions,
associating a name with a type. Finally, the varinfo represents variable
definitions. 
All variables in a program are indexed: within a file each variable has a unique
identifier, called {\tt vid}. Global variable identifiers are cross referenced
across files via the \_gxrefs.xml files. The address? element of the varinfo
indicates whether the address of the variable has been taken (that is, if it can
be aliased). The param element is a positive integer for formal parameters of
functions, indicating its sequence number in the list of parameters (starting at
1), or zero for all other variables.

\begin{table}
\centering
\small
\begin{tabular}{l|l|c||c|c|c|c|c|c|c|c|c|l}
&\multicolumn{2}{c||}{tags} & \multicolumn{9}{|c|}{args}   \\ \hline
&\multicolumn{1}{c|}{1} & 2 & 1 & 2 & 3 & 4 & 5 & 6 & 7 & 8 & 9 \\ \hline 
\multicolumn{12}{l}{compinfo} \\ \hline
& name & & key & struct? & {\tt attrs} & \multicolumn{6}{|l|}{*{\tt fieldinfo}} \\ \hline
\multicolumn{12}{l}{enuminfo} \\ \hline
& name & ekind & {\tt attrs} & \multicolumn{8}{|l|}{*{\tt enumitem}} \\ \hline
\multicolumn{12}{l}{enumitem} \\ \hline
& name & & {\tt exp} & {\tt loc} & & & & & & & \\ \hline
\multicolumn{12}{l}{fieldinfo} \\ \hline
& name & & key & {\tt typ} & bitfield & {\tt attrs} & {\tt loc} & & & & \\ \hline
\multicolumn{12}{l}{init} \\ \hline
& "single" & & {\tt exp} & \multicolumn{8}{|l|}{\mbox{}} \\
& "compound" & & {\tt typ} & \multicolumn{8}{|l|}{*{\tt offset-init}} \\ \hline
\multicolumn{12}{l}{loc} \\ \hline
& & & {\tt filename} & byte & line & & & & & &\\ \hline
\multicolumn{12}{l}{offset-init} \\ \hline
& & & {\tt offset} & {\tt init} & & & & & & & \\ \hline
\multicolumn{12}{l}{typeinfo} \\ \hline
& name & & {\tt typ} & & & & & & & & \\ \hline
\multicolumn{12}{l}{varinfo} \\ \hline
& name & storage & vid & {\tt typ} & {\tt attrs} & global? & inlined? & {\tt loc} &
   address? & param & {\tt init} \\ \hline
&\multicolumn{1}{c|}{1} & 2 & 1 & 2 & 3 & 4 & 5 & 6 & 7 & 8 & 9 \\ \hline 
\end{tabular}
\caption{\label{tab:decls}Declarations tables: tags and arguments for program declarations}
\end{table}


%\begin{center}
%\begin{figure}[h]
%\centering
%\includegraphics[width=.85\textwidth]{figures/initialdisplay.png}
%\caption{\label{fig:initialdisplay}Initial display (textcrunchr\_1)}
%\end{figure}
%\end{center}

\end{document}
